\chapter{Desarrollo del Proyecto}

\minitoc

\section{Introducción}

%% Decir qué cuenta esta sección y los pasos generales que se han dado para
%% desarrollar el \emph{Process Editor}

El presente capítulo detalla el proceso de desarrollo del proyecto realizado dentro de este Trabajo Fin de Grado. Tal como se ha comentado, el objetivo del proyecto era .. 

Para alcanzar este objetivo, se siguió el esquema de trabajo que se muestra en la Figura~\ref{}.  

En la primera etapa ...

Las siguientes secciones proporcionan detalles sobre la ejecución de cada paso.

\section{Desarrollo del Editor Gráfico}

%% Describir los pasos, a nivel general, para crear el editor gráfico sencillo. 

%% Añadir algo de código. 

%% Qué se hizo en otras iteraciones, a modo de la lista de la compra (y teniendo en 
%%    cuenta que son cosas gráficas)

%% Código de aquello en lo que te quieras lucir (opcional)

%% Hablar algo de pruebas. 

\section{Integración el editor con \emph{Vaadin}}

%% Explicar que necesitamos hacer que los eventos del editor se redirijan a Vaadin

%% Explicar que la redirección se hace mediante un conector en Javascript

%% Mostrar un trozo de código del conector muy sencillo

%% Explicar que en algunos casos fue necesario alterar el flujo de funcionamiento 
%% normal de GoJS y describir el caso del link.

%% Hablar algo de pruebas. 

\section{Integración el editor con LUCA}

%% Decribir los pasos, a nivel general, para integrar el editor básico. 

%% Mostrar captura de la interfaz

%% Qué se hizo en otras iteraciones, a modo de la lista de la compra.

%% Mostrar código de algo no trivial 

%% Hablar algo de pruebas. 

\section{Gestión y Persistencia de los Procesos}

El objetivo de este paso era que LUCA pudiese almacenar y gestionar procesos. Para ello lo primero era definir un modelo conceptual de datos para los procesos. Dicho modelo conceptual se muestra en la Figura~\ref{}.

%% Explicar que ese modelo concetual de datos se implementa en Java y con JPA se 
%% genera el esquema relacional y el puente objeto-relacional.

%% Indicar qué ha sido necesario modificar en la capa de negocio (control de accesos)
%% y negocio (servicio). Mostrar ejemplos de código

%% Qué se hizo en otras iteraciones, a modo de la lista de la compra.

%% Hablar algo de pruebas, y si las hay, mostrar código.
