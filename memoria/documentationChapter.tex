 \chapter{Antecedentes}

Este capítulo describe el proceso de adquisición de conocimientos necesarios para poder llevar a cabo el proceso de diseño arquitectónico y de construcción o implementacíon de la aplicación. De esta fomra se puede realizar una planificación mas cercana a la realidad y partir de unos conocimientos mínimos para empezar a elaborar el proyecto.
	
	 \minitoc
	
\section{GO.JS}

%%====================================================================================================
%% NOTE(Pablo): Nunca se ponen los logos de la herramienta. Queda fatal, parece que quieres rellenar
%%   espacio por rellenar. No pongas tantas subsecciones.
%%====================================================================================================
	 		
Go.JS \cite{gojs} es una biblioteca de JavaScript para implementar editores gráficos dentro de interfaces web. GoJS facilita la implementación de funciones tales como definición de símbolos gráficos, gestión de paletas de símbolos, arrastrar y soltar (\emph{drag and drop}), copiar y pegar, edición de etiquetas texto asociadas a símbolos gráficos, menús contextuales, función de deshacer o gestión de eventos, entre muchas otras funcionalidades.
	 		
%%====================================================================================================
%% NOTE(Pablo): Ponme un ejemplo chorra de cómo se dibuja y mueve un círculo en Go.JS y descríbelo
%%====================================================================================================

%%========================================================================%%
%% NOTA(Pablo): Cuando esté listo lo de Go.JS sigo con el resto           %%
%%========================================================================%%

\section{Vaadin}
	 	
	 		\begin{figure}[H]
	 			\centering
	 			\includegraphics[scale=1]{Vaadin-logo.png}
	 			\caption{Vaadin Logo}\label{fig:Vaadin-logo}
	 		\end{figure}
 		
 			\subsection{¿Qué es GO.JS?}
 				Vaadin\cite{vaadin} es un framework de desarrollo de SPA que permite escribir el código de dichas aplicaciones en Java o en cualquier otro lenguaje soportado por la JVM 1.6+. Esto permite la programación de la interfaz gráfica en lenguajes como Java 8, Scala o Groovy, por ejemplo.
 			
 			\subsection{Características}
 				Uno de las características diferenciadores de Vaadin es que, contrario a las librerías y frameworks de JavaScript típicas, presenta una arquitectura centrada en el servidor, lo que implica que la mayoría de la lógica es ejecutada en los servidores remotos. Del lado del cliente, Vaadin está construido encima de Google Web Toolkit, con el que puede extenderse.
 	
 			\subsection{Aplicación}
 				En este proyecto, Vaadin se encargara de realizar la comunicación entre el cliente y el servidor. De esta forma, será capaz de enviar y recibir datos, eventos y peticiones entre el componente Javascript (cliente) y el servidor.
 			
 			
	 			\begin{figure}[H]
	 				\centering
	 				\includegraphics[scale=1.5]{schema.png}
	 				\caption{Esquema Cliente-Servidor}\label{fig:schema}
	 			\end{figure}
 			
	
	

	
	