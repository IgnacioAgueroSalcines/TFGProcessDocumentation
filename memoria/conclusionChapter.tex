\chapter{Sumario, Conclusiones y Trabajos Futuros}

%%===============================================================================%%
%% NOTe(Pablo): Esto distribuyelo como veas por las nuevas secciones             %%
%%===============================================================================%%

\section{Sumario}

%% Repasas lo que ha descrito el documento.
El trabajo actual, enmarcado dentro del proyecto LUCA, ha descrito el objetivo de LUCA de obtener una interfaz de aceso uniforme a fuentes de datos heterogéneas. Se ha explicado la arquitectura y funcionamiento de LUCA antes y después de realizar este proyecto, y se ha expuesto el alcance del proyecto.

En un apartado de documentación se ha explicado la herramienta de \emph{GoJS}, el framework de \emph{Vaadin} y el patrón \emph{Model-View-Presenter}. En la etapa de desarrollo se ha explicado el flujo iterativo de construcción del proyecto, pasando por la construcción del editor gráfico, y su integración con \emph{Vaadin} y LUCA, además, del desarrollo de la gestión y persistencia de los \emph{procesos}.

\section{Resultados}

%% Lista de cosas que hace tu proyecto
El proyecto desarrollado, el cual incorpora al actual LUCA la capacidad de gestionar \emph{procesos}, proporciona las siguientes características:
\begin{itemize}
	\item Gestionar y ejecutar procesos de forma gráfica.
	\item Comprobar restricciones en los enlaces entre consultas.
	\item Visualizar las consultas, así como, información de las mismas y de sus enlaces.
	\item Dar información parcial durante la ejecución.
	\item Exportar los resultados de los procesos.
\end{itemize}

\section{Conclusiones}

%% Experiencia propia
La experiencia durante el desarrollo del proyecto ha sido más que satisfactoria, ya que los conocimiento que he aprendido a lo largo del grado, me han sido indispensables para poder llevarlo a cabo, además, debido al uso de nuevos frameworks y herramientas, he sido capaz de ampliar mis conocimientos en \emph{Vaadin} y en \emph{GoJS}.

La oportunidad de llevar a cabo el proyecto en la empresa CIC Consulting Informático, me ha permitido adquirir un hábito de trabajo, así como de responsabilidades y fijaciones a horarios que no se puede aprender de ninguna otra forma. Además, el apoyo recibido por el equipo de LUCA ha sido excepcional y he podido aprender mucho con ellos.

\section{Trabajos Futuros}

%% Qué se va a hacer en el futuro, tanto en LUCa como en la parte de tu proyecto
Con el proyecto integrado en LUCA, se espera poder realizar ampliaciones de sucesivas versiones del producto tales como la capacidad de concatenar procesos (tal y como se utilizan consultas, poder utilizar los procesos ya creados), o crear consultas desde una consola gráfica, utilizando el editor gráfico, de forma que se agilice el actual proceso de formularios para poder construir una.


	