\appendix
\chapter{Tratamiento de eventos de la clase Diagram Component}\label{aped.A}
A continuación se muestran el resto de eventos que son tratado en el ejemplo del editor gráfico y su integración con Vaadin.

\begin{figure}[H]
	\centering
	\begin{lstlisting}[language=Javascript]
	addFunction("SelectionMoved", new JavaScriptFunction()
	{
	
		@Override
		public void call(JsonArray arguments)
		{
		
			List<Node> nodes = getState().getNodes();
			for (Node node : nodes)
			{
				node.update();
			}
			System.out.println("SelectionMoved at " + arguments.toJson());
		}
	});\end{lstlisting}
	\caption{Evento de elemento movido}
	\label{fig:selectionMoved}
\end{figure}


\begin{figure}[H]
	\centering
	\begin{lstlisting}[language=Javascript]
	addFunction("TextEdited", new JavaScriptFunction()
	{
	
		@Override
		public void call(JsonArray arguments)
		{
		
			Map<String, String> map = new HashMap<>();
			if(arguments != null)
			{
				try
				{
					String srt = arguments.toJson();
					map = JsonUtils.jsonToMap(
					srt.substring(1, srt.length() - 1));
				}
				catch (JSONException e)
				{
					e.printStackTrace();
				}
			}
			List<Node> nodes = getState().getNodes();
			for (Node node : nodes)
			{
				if(map.get("oldValue").equals(node.getName()))
				{
					node.setName(map.get("newValue"));
				}	
			}
			System.out.println("TextEdited at " + arguments.toJson());
		}
	});\end{lstlisting}
	\caption{Evento de edición de texto}
	\label{fig:textEdited}
\end{figure}


\begin{figure}[H]
	\centering
	\begin{lstlisting}[language=Javascript]
	addFunction("ObjectDoubleClicked", new JavaScriptFunction()
	{
	
		@Override
		public void call(JsonArray arguments)
		{
		
			System.out.println("ObjectDoubleClicked at " 
			+ arguments.toJson());
		}
	});\end{lstlisting}
	\caption{Evento de doble selección}
	\label{fig:doubleClick}
\end{figure}
