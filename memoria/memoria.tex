%clase del documento
\documentclass[a4paper,12pt]{book}

%paquete de idioma y codificación de caracteres
\usepackage[spanish]{babel}
\usepackage[utf8]{inputenc}
\usepackage{afterpage}

%figuras en un lugar determinado
\usepackage{float}

\usepackage{cite} % para contraer referencias

\newcommand\tab[1][1cm] %tabulaciones


% soporte gráfico
\usepackage{graphicx} % figuras
\usepackage{subfigure} % subfiguras
\graphicspath{ {images/} }

%indice dentro de secciones
\usepackage{minitoc}

%datos del documento
\author{Ignacio Agüero Salcines}
\title{Especificación Gráfica de Procesos de Recuperación de Datos en LUCA}

\setcounter{tocdepth}{5}
\setcounter{secnumdepth}{5}

%Listing Package
%Define the listing package
\usepackage{listings} %code highlighter
\usepackage{color} %use color
\definecolor{mygreen}{rgb}{0,0.6,0}
\definecolor{mygray}{rgb}{0.5,0.5,0.5}
\definecolor{mymauve}{rgb}{0.58,0,0.82}

%Customize a bit the look
\lstset{ %
	backgroundcolor=\color{white}, % choose the background color; you must add \usepackage{color} or \usepackage{xcolor}
	basicstyle=\footnotesize, % the size of the fonts that are used for the code
	breakatwhitespace=false, % sets if automatic breaks should only happen at whitespace
	breaklines=true, % sets automatic line breaking
	captionpos=b, % sets the caption-position to bottom
	commentstyle=\color{mygreen}, % comment style
	deletekeywords={...}, % if you want to delete keywords from the given language
	escapeinside={\%*}{*)}, % if you want to add LaTeX within your code
	extendedchars=true, % lets you use non-ASCII characters; for 8-bits encodings only, does not work with UTF-8
	frame=single, % adds a frame around the code
	keepspaces=true, % keeps spaces in text, useful for keeping indentation of code (possibly needs columns=flexible)
	keywordstyle=\color{blue}, % keyword style
	% language=Octave, % the language of the code
	morekeywords={*,...}, % if you want to add more keywords to the set
	numbers=left, % where to put the line-numbers; possible values are (none, left, right)
	numbersep=5pt, % how far the line-numbers are from the code
	numberstyle=\tiny\color{mygray}, % the style that is used for the line-numbers
	rulecolor=\color{black}, % if not set, the frame-color may be changed on line-breaks within not-black text (e.g. comments (green here))
	showspaces=false, % show spaces everywhere adding particular underscores; it overrides 'showstringspaces'
	showstringspaces=false, % underline spaces within strings only
	showtabs=false, % show tabs within strings adding particular underscores
	stepnumber=1, % the step between two line-numbers. If it's 1, each line will be numbered
	stringstyle=\color{mymauve}, % string literal style
	tabsize=2, % sets default tabsize to 2 spaces
	title=\lstname % show the filename of files included with \lstinputlisting; also try caption instead of title
}
%END of listing package%

\definecolor{darkgray}{rgb}{.4,.4,.4}
\definecolor{purple}{rgb}{0.65, 0.12, 0.82}

%define Javascript language
\lstdefinelanguage{JavaScript}{
	keywords={typeof, new, true, false, catch, function, return, null, catch, switch, var, if, in, while, do, else, case, break},
	keywordstyle=\color{blue}\bfseries,
	ndkeywords={class, export, boolean, throw, implements, import, this},
	ndkeywordstyle=\color{darkgray}\bfseries,
	identifierstyle=\color{black},
	sensitive=false,
	comment=[l]{//},
	morecomment=[s]{/*}{*/},
	commentstyle=\color{purple}\ttfamily,
	stringstyle=\color{red}\ttfamily,
	morestring=[b]',
	morestring=[b]"
}

\lstset{
	language=JavaScript,
	extendedchars=true,
	basicstyle=\footnotesize\ttfamily,
	showstringspaces=false,
	showspaces=false,
	numbers=left,
	numberstyle=\footnotesize,
	numbersep=9pt,
	tabsize=2,
	breaklines=true,
	showtabs=false,
	captionpos=b
}
%End Listing



\begin{document}
	
	
	\pagestyle{empty}
	\dominitoc% Inicializacion
	\tableofcontents
	\cleardoublepage
	
	\pagestyle{plain}
	
	\listoffigures
	\listoftables
	\thispagestyle{empty}
	\cleardoublepage
	
	\pagenumbering{roman}
	\chapter*{Agradecimientos}
	Me gustaría dar agradecimientos a mi familia y facultad, ya que sin ellos esto no habría sido posible nada de estos.
	
	\vspace{5mm}
	
	\noindent Es importante agradecer también a CIC Consulting Informático por permitirme la oportunidad de realizar el desarrollo del proyecto en su empresa, sin olvidarme de mis compañeros de LUCA, que han sido un gran apoyo durante el mismo..
	
	\vspace{5mm}
	
	\noindent Para finalizar, me gustaría gradecer a mi mentor Pablo, por guiarme durante el desarrollo del proyecto con eficacia y ayudarme a afrontar este trabajo de fin de grado.
	\cleardoublepage
	
	\clearpage
	
	\chapter*{Resumen}
	Las empresas actuales no utilizan un único sistema de información que de soporte a sus procesos de trabajo, sino un  ecosistema de sistemas información que dan soporte a diferentes procesos de negocio ejecutados dentro de dicha organización. Como consecuencia de esta nueva situación, cuando un usuario	quiere obtener una información concreta cuyos datos residen en varios de estos
	sistemas, necesita acceder a cada uno de estos sistemas, extraer de cada sistema la información que precisa, filtrarla y unificarla para finalmente	obtener los datos requeridos.
	
	\vspace{5mm}
	
	Por ejemplo, una tienda de electrodomésticos podría tener sistemas informáticos diferentes para el departamento de atención al cliente, para el departamento técnico de postventa y para el departamento de compras y adquisiciones.Por tanto, para conocer el estado actual de una reparación, podríamos necesitar:
		\begin{itemize}
			\item  Acceder al primer sistema para obtener el identificador de la incidencia y en qué fase de su gestión se encuentra.
			\item  Comprobado que la incidencia está actualmente en reparación, recuperaríamos otro sistema el estado detallado de la reparación, comprobando que está a la espera de una pieza.
			\item Finalmente accederíamos al sistema de compra y adquisiciones para comprobar cuando está prevista la entrega de dicha pieza. Los sistemas de almacenamiento de la información pueden ser diversos, incluyendo desde un servicio web, una base de datos relacional, un repositorio de ficheros accesible vía FTP o una base de datos NoSQL.
		\end{itemize}
	
	\vspace{5mm}
	
	El objetivo de este proyecto es facilitar dicho proceso de composición al usuario mediante el desarrollo de un mecanismo gráfico para la especificación de estos procesos de composición de consultas.
	

	\vspace{5mm}
	
	\textbf{Palabras clave}: 
	\cleardoublepage
	
	\clearpage
	
	\chapter*{Preface}

	

	\textbf{Keywords}:
	\cleardoublepage
	
	\pagenumbering{arabic}
	\setcounter{page}{1}
	
	\clearpage
	
	\chapter{Introducción}

\minitoc
	
\section{Introducción}

%% Aquí hay que añadir un poco de paja. 

El presente trabajo se enmarca dentro del proyecto LUCA. Por tanto, para poder comprender los objetivos de este Trabajo Fin de Grado, se hace necesario describir primero dicho proyecto, lo cual se realiza en la siguiente sección.

\section{LUCA}

\subsection{Motivación}

En los últimos años, el volumen de datos recogidos y manipulados por las empresas ha aumentado de forma vertiginosa. Estos datos se han ido almacenando en diferentes tipos de fuentes conforme las empresas crecían y sus sistemas evolucionaban y se fusionaban. Como resultado de  este proceso no es extraño actualmente encontrar empresas que tengan sus datos almacenados en sistemas tan dispares como bases de datos relacionales, hojas XML o repositorios FTP.

Como consecuencia de esta nueva situación, cuando un usuario quiere obtener una información concreta cuyos datos residen en varios de estos sistemas, éste necesita acceder a cada uno de estos sistemas, extraer de cada sistema la información que precisa, y finalmente filtrarla y unificarla para finalmente obtener los datos requeridos.

Por ejemplo, una cadena de venta de electrodomésticos podría tener sistemas informáticos diferentes para el departamento de atención al cliente, para el departamento técnico de postventa y para el departamento de compras y adquisiciones. Por tanto, para conocer con precisión el estado actual de una reparación, podríamos necesitar:

\begin{enumerate}
	\item Acceder al sistema de atención al cliente para obtener el identificador de la incidencia y en qué fase de su gestión se encuentra.
	\item Una vez corroborado que la incidencia está actualmente siendo atendida, recuperaríamos del sistema de gestión de reparaciones el estado detallado de la reparación. Como resultado de esta operación, supongamos que averiguamos que la reparación está a la espera de recibir una pieza que se ha de sustituir.
	\item Finalmente, para poder hacer una estimación de cuando podría estar lista la reparación, accederíamos al sistema de compra y adquisiciones para averiguar cuando está prevista la entrega de la pieza solicitada.
\end{enumerate}

Como hemos comentado anteriormente, a cada uno de estos sistemas podría accederse de manera diferente. Por ejemplo, el primero podría consultarse utilizando un servicio web. La información del segundo podría recuperarse accediendo directamente a una base de datos relacional, mientras que la información del tercero se obtendría analizando órdenes de compra en formato \emph{pdf} almacenadas en un repositorio de ficheros compartido. Por tanto, el usuario, para poder realizar este proceso, necesita conocer las particularidades de cada sistema y de su forma de acceso.

Para aliviar esta situación, dentro de la empresa CIC, se está desarrollando una aplicación denominada LUCA, a la cual contribuye este Trabajo Fin de Grado. Para facilitar este proceso de recuperación de información, LUCA proporciona un lenguaje común para todas las fuentes de datos a unificar, permitiendo al usuario abstraerse de los detalles de cada fuente.

\subsection{Funcionamiento de LUCA}

A continuación, se detalla brevemente el funcionamiento de \emph{LUCA}. Para ello, utilizaremos como ejemplo una consulta a la base de datos de LUCA, esta consulta obtendrá los procesos en función de un estado (los estados pueden ser en edición, publicado, es decir, preparada para ser ejecutada, o borrado).

\begin{figure}[!tb]
    \centering
 	\includegraphics[width=\linewidth]{capturasLuca/menuLuca.png}
	\caption{Menu Principal LUCA}
    \label{fig:menuLuca}
\end{figure}

En el menu principal de LUCA, nos aparece una vista con un conjunto de consultas que el usuario ha marcado como favoritas (Figura~\ref{fig:menuLuca}), es decir, que suele utilizar con frecuencia. Como se dijo anteriormente, se utilizará la consulta de los procesos por estado.

\begin{figure}[!tb]
	\centering
	\includegraphics[width=\linewidth]{capturasLuca/gestionConsultas.png}
	\caption{Vista de Gestión de Consultas}
	\label{fig:gestionConsultas}
\end{figure}

En la vista de gestión de consultas (Figura~\ref{fig:gestionConsultas}) se pueden realizar las operaciones propias de la gestión, como es crear, modificar, eliminar o ejecutar entre otras.

No obstante, para que dichas consultas pueden ser ejecutadas, es necesario un usuario con conocimientos suficientes para ello, al que denominaremos en adelante el \emph{creador de consultas}.

\begin{figure}[!tb]
	\centering
	\includegraphics[width=\linewidth]{capturasLuca/creacionConsulta.png}
	\caption{Creación de una Consulta}
	\label{fig:creacionConsulta}
\end{figure}

Para poder realizar esta tarea, el creador de consultas accedería a la interfaz dedicada a esta tarea (ver Figura~\ref{fig:creacionConsulta}). En esta interfaz, definiría primero las variables de entrada y salida de la consulta (Figura~\ref{fig:creacionConsulta}, etiqueta 1). A continuación, especificaría cómo llevar a cabo dicha consulta a bajo nivel. Para ello puede utilizar una serie de facilidades y primitivas proporcionadas por LUCA. Para la consulta ejemplo, utilizando estas facilidades, se especifica que se desea obtener el nombre, la descripción, el nombre de la tarea asíncrona y el usuario, y se establece como variable de entrada el estado del proceso.

Tras guardar la consulta se puede ejecutar directamente desde la ventana de creación, sin embargo, si la consulta ya ha sido ejecutada previamente y el usuario la ha publicado (tras ejecutar una consulta desde la ventana de creación, si se ha ejecutado correctamente se le permite al usuario cambiar su estado de edición a publicada, esto significa que la consulta está preparada para ser ejecutada), el usuario puede ir a la ventada de ejecución de consultas, y desde ahí ejecutarla.

La principal ventaja que aporta LUCA es que el proceso de ejecución de consultas es opaco para el usuario que la ejecuta. El usuario sólo tiene que proporcionar los parámetros necesarios de entrada y seleccionar un formato de salida. Por tanto, el proceso de ejecución de consultas es exactamente el mismo con independencia de la fuente a la cual se accede.

Por ejemplo, en el caso de la consulta \emph{Get Procesos Por Estado} sería necesario proporcionar el estado de los procesos de los que queremos obtener información. Una vez introducido dicho estado, se seleccionaría el botón de ejecución de la consulta (Figura~\ref{fig:ejecucionConsulta}, etiqueta 2). Finalmente, se muestra el resultado de la consulta, el cuál puede ser visualizado de diferentes formas en función del recurso al que se llama. En nuestro caso, se muestra como una tabla ya que es una consulta a base de datos (Figura~\ref{fig:ejecucionConsulta}, etiqueta 3).

Lo importante de este proceso es que, una vez definida la consulta, ésta se puede ejecutar fácilmente sin conocer los detalles internos de la misma, incluso hasta el tipo de sistema al que se accede.

	\begin{figure}[!tb]
		\centering
		\includegraphics[width=\linewidth]{capturasLuca/ejecucionConsulta.png}
		\caption{Vista de Ejecución de Consultas}\label{fig:ejecucionConsultas}
	\end{figure}

\subsection{Limitaciones actuales de LUCA}

Actualmente, LUCA proporciona mecanismos para permitir al usuario recuperar de manera uniforme información de diferentes fuentes de datos. No obstante, LUCA por el momento sólo es capaz recuperar información de una única fuente de datos a la vez. Por tanto, cuando es necesario combinar información procedente de distintas fuentes, el propio usuario es el que debe realizar dicho proceso de composición a mano, ejecutando él cada consulta, y utilizando las salidas de cada una de ellas como entradas para las siguientes.


Un ejemplo de dicho proceso de composición sería la necesidad de un dependiente de una tienda de electrodomésticos de obtener la edad de los usuarios que compraron lavadoras durante el mes pasado. Actualmente, la secuencia de consultas que debería de realizar serían las siguientes:

\begin{itemize}
	\item Primero necesitaría obtener el registro de compras del mes pasado del sistema.
	\item Después, tras guardar dicho registro, tendría que, uno por uno, seleccionar los que se corresponden con lavadoras.
	\item Una vez que el usuario tiene las lavadoras compradas el mes pasado, éste tendría que extraer que usuarios han comprado las lavadoras.
	\item Por último, debería buscar en el sistema cada usuario que ha realizado la compra, a partir del nombre obtenido en el punto anterior, y anotar su edad.
\end{itemize}

En adelante, estas cadenas de consultas para obtener un resultado concreto las denominaremos \emph{procesos}. El problema actual de LUCA, tal como ilustra el ejemplo anterior, es que no soporta el concepto de \emph{proceso}. Por tanto, para ejecutar un proceso,  el usuario tiene que realizar una larga y compleja secuencia de acciones.

\section{Objetivos del Trabajo de Fin de Grado}

El objetivo general de este Trabajo Fin de Grado es integrar en LUCA el concepto de \emph{proceso}. Para ello, hay que dar soporte a dos cuestiones diferentes: (1) la ejecución de los procesos; y (2) la especificación de procesos. Por tanto, el objetivo general de este trabajo se descompone en estos dos subobjetivos principales.

El primer objetivo implica poder tratar procesos en LUCA de la misma forma que se trata las consultas. Es decir, los procesos deberán aparecer como en las consultas bajo una pestaña de gestión y otra de ejecución. Obviamente, la complejidad de ejecutar una proceso es mayor que la de ejecutar una consulta, ya que necesitamos ejecutar varias consultas, guardar resultados intermedios y utilizar estos resultados como entradas para otras consultas.

El segundo objetivo, que es el que implica una mayor complejidad, consiste en facilitar la especificación de procesos en LUCA. Para que un proceso pueda ser ejecutado, primero debe ser especificado, indicando qué consultas lo componen y cómo se relacionan. De acuerdo con los deseos expresados por los responsables del proyecto LUCA y la empresa CIC, dicho mecanismo de especificación debía ser gráfico, permitiendo así componer consultas de manera visual mediante la interconexión de las salidas de unas con las entradas de otras.

\section{Alcance del Proyecto}

Para refinar estos dos grandes objetivos en una serie de requisitos más concretos, se llevó a cabo en primer lugar una reunión con el Jefe y el Gerente del proyecto. El objetivo de dicha reunión era conocer LUCA en profundidad. A continuación, dado que la fase de Ingeniería de Requisitos para este proyecto ya había sido realizada por la propia empresa, se nos proporcionaron unos documentos técnicos con los requisitos técnicos tanto para la ejecución de procesos como para el desarrollo del componente gráfico de especificación de procesos. Estos documentos pueden encontrarse en el Anexo adjunto a la memoria.

%%==============================================================================%%
%% NOTE(Pablo): Lo de abajo se quita por simplicidad                            %%
%%==============================================================================%%
%%
%% Como ya se ha mencionado, en estos documentos se pueden encontrar los
%% requisitos técnicos atribuidos al proyecto, pero, de forma resumida, se
%% centran en tres pilares o requisitos principales:
%%
%% \begin{itemize}
%%  	\item Concatenación de las consultas entre si pertenecientes a un
%%            mismo proceso.
%% 	\item Visualización del progreso de ejecución del proceso.
%% 	\item Aplicar criterios de navegación a partir de los resultados de salidas.
%% \end{itemize}
%%
%%
%%==============================================================================%%
%%
%% Comentar que la arquitectura ya estaba definida, por lo que el proyecto se reduce
%% a las fases de diseño detallado, implementación y pruebas.
%%
%% El proyecto genera como resultado dos módulos grande y bien diferenciados: 
%%    Process Editor Component y (se va a cambiar). 
%%==============================================================================%%

\section{Sumario}

%% To be written
	
	\clearpage
	
	 \chapter{Antecedentes}

Este capítulo describe el proceso de adquisición de conocimientos necesarios para poder llevar a cabo el proceso de diseño arquitectónico y de construcción o implementacíon de la aplicación. De esta fomra se puede realizar una planificación mas cercana a la realidad y partir de unos conocimientos mínimos para empezar a elaborar el proyecto.
	
	 \minitoc
	
\section{GO.JS}

	%%====================================================================================================
%% NOTE(Pablo): Nunca se ponen los logos de la herramienta. Queda fatal, parece que quieres rellenar
%%   espacio por rellenar. No pongas tantas subsecciones.
%%====================================================================================================
	 		
Go.JS \cite{gojs} es una biblioteca de JavaScript para implementar editores gráficos dentro de interfaces web. GoJS facilita la implementación de funciones tales como definición de símbolos gráficos, gestión de paletas de símbolos, arrastrar y soltar (\emph{drag and drop}), copiar y pegar, edición de etiquetas texto asociadas a símbolos gráficos, menús contextuales, función de deshacer o gestión de eventos, entre muchas otras funcionalidades.
	 		
%%====================================================================================================
%% NOTE(Pablo): Ponme un ejemplo chorra de cómo se dibuja y mueve un círculo en Go.JS y descríbelo
%%====================================================================================================


\section{Vaadin}
 		
 	

\emph{Vaadin}~\cite{vaadin} es un \emph{framework} para el desarrollo de aplicaciones \emph{web} avanzadas, también conocidas como \emph{Rich-Internet Applications (RIA)}~\cite{ria}. El objetivo del paradigma \emph{RIA} es desarrollar aplicaciones \emph{web} con interfaces avanzadas que les haga asemejarse a las aplicaciones de escritorio. La principal ventaja que aporta \emph{Vaadin} es que permite escribir aplicaciones en código Java, como si fuesen de escritorio, y luego este código es transformado para que funcione en tecnologías web como HTML (\emph{HyperText Markup Language})~\cite{html}, CSS (\emph{Cascading Style Sheets})~\cite{css}, Javascript~\cite{javascript}, HTTP (\emph{Hypertext Transfer Protocol})~\cite{http} o AJAX (\emph{Asynchronous JavaScript and XML})~\cite{ajax}.

Una de las características diferenciadores de \emph{Vaadin} es que, al contrario de las librerías de JavaScript tradicionales, \emph{Vaadin} también contempla la parte del servidor, por lo se generan tanto las llamadas al servidor desde la interfaz gráfica (\emph{front-end}) como la recepción y tratamiento de esas llamadas en la parte del servidor (\emph{back-end}).

Para abstraer al usuario de elementos relacionados con HTML o Javascript, Vaadin utiliza los llamados \emph{componentes}. Un componente representa un elemento gráfico o \emph{widget}. Para el desarrollo de los componentes, Vaadin proporciona una serie de clases reutilizables que contienen los infraestructura necesaria para facilitar su traducción a código HTML y Javascript. Para crear \emph{componentes}, los desarrolladores de Vaddin deben simplemente extender estas clases.



\subsection{Arquitectura Vaadin}

\begin{figure}
	\centering
	\includegraphics[width=\linewidth]{vaadin/vaadinArch.png}
	\caption{Esquema Modelo-Vista-Presentador en Vaadin}
	\label{fig:vaadinMVP}
\end{figure}

El framework de Vaadin\cite{vaadinArch} separa la comunicación en dos módulos (Figura~\ref{fig:vaadinMVP}): El lado cliente y el lado servidor. El lado servidor se caracteriza por poseer toda la carga lógica de la aplicación, además de encargarse de renderizar la interfaz comunicándose con el lado cliente utilizando \emph{Ajax}\cite{ajax}. El lado cliente es el encargado de traducir todos los ficheros \emph{Java} a ficheros \emph{Javascript} que puedan ser ejecutados por un navegador y mostrados en la interfaz.

Ambos módulos se comunican entre si a partir de \emph{APIs}\cite{api} propias, y tienen un modelo de componentes y elementos gráficos copia en ambos lados. De este modo si se produce un evento sobre la interfaz, el lado cliente se comunica con la \emph{API} servidora para notificarla los eventos y/o los cambios en los componentes, una vez recibida y tratada la llamada en el lado servidor, este se encarga de conectarse a la \emph{API} cliente junto con los nuevos componentes gráficos, es decir, la nueva vista, para que el lado cliente la muestre por pantalla.

A raíz de este comportamiento, Vaadin suele estar ligado al patrón \emph{MVP}\cite{mvp} (explicado en el apartado de la arquitectura de LUCA) debido a que aísla la vista del modelo y las conecta a través de un intermediario que es el presentador, este se encarga de comunicarse con ambas y de ejecutar la lógica de negocio de la aplicación. De este modo cuando en la vista se produce un cambio, es notificado en el presenter, este lo gestiona (comunicándose con el modelo si es necesario) y después manda a la vista recargarse con los cambios producidos. Se puede apreciar que siguiendo este patrón, el conjunto de componentes que proporciona la vista para mostrar por pantalla estarían de forma duplicada tanto en la parte servidora como cliente a la espera de recibir eventos o ser recargadas.


A continuación, se explica el funcionamiento de Vaadin mediante la construcción, a modo de ejemplo, de un árbol de tareas propio de la gestión de proyectos  (Figura~\ref{fig:vaadinExampleImage}).

\begin{figure}[!tb]
	\centering
	\includegraphics[width=\linewidth]{vaadinExampleImage.png}
	\caption{Árbol de Proyectos}
	\label{fig:vaadinExampleImage}
\end{figure}

\begin{figure}[!tb]
	\centering
	\begin{lstlisting}[language=Java]
	@Override
	protected void init(VaadinRequest request) {
	
	final VerticalLayout layout = new VerticalLayout();
	layout.setSpacing(true);
	layout.setMargin(true);
	
	final TreeGrid grid = new TreeGrid();
	grid.setWidth(800, Unit.PIXELS);
	grid.setHeight(450, Unit.PIXELS);
	
	JobContainer container = new JobContainer();
	grid.setContainerDataSource(container);
	
	layout.addComponent(grid);
	setContent(layout);
	}
	\end{lstlisting}
	\vspace{-15pt}
	\caption{Interfaz de Usuario Vaadin}
	\label{fig:uiVaadin}
\end{figure}

Para construir este ejemplo, en primer lugar definimos su interfaz gráfica. Para crear dicha interfaz gráfica, nos basamos en un componente gráfico, o \emph{widget}, denominado \emph{TreeGrid}  (Figura~\ref{fig:jobContainer}, Línea 8). Como puede observarse, este componente gráfico se usa directamente desde código Java, tal como se crearía una interfaz Java de escritorio, utilizando elementos propios de Java como los \emph{layouts} (Figura~\ref{fig:uiVaadin}, Líneas~4\-6) y no siendo necesario escribir nada en Javascript o HTML. Este componente mostrará los datos proporcionados por el contenedor de datos \emph{JobContainer} (Figura~\ref{fig:uiVaadin}, Líneas~12-13).

La clase \emph{JobContainer} (Figura~\ref{fig:jobContainer}) es la que proporcionará los datos que se muestran en el \emph{grid}. Esta clase extiende de una clase de Vaadin llamada \emph{HierarchicalContainer} y se encarga de implementar toda la lógica para almacenar de forma jerárquica los nodos. Además, implementa las interfaces de Vaadin \emph{Collapsible} (Figura~\ref{fig:jobContainerCollapsible}) y \emph{Measurable} (Figura~\ref{fig:jobContainerMeasurable}), encargadas de contraer el árbol de elementos y de calcular la profundidad del elemento en la jerarquía, respectivamente.

\begin{figure}[!tb]
	\centering
	\begin{lstlisting}[language=Java]
	public class JobContainer extends HierarchicalContainer 
                               implements Collapsible, Measurable {
	
		static final String PROPERTY_NAME = "Name";
		static final String PROPERTY_HOURS = "Hours done";
		static final String PROPERTY_MODIFIED = "Last modified";
		
		public JobContainer() {
			addContainerProperty(PROPERTY_NAME, String.class, "");
			addContainerProperty(PROPERTY_HOURS, Integer.class, 0);
			addContainerProperty(PROPERTY_MODIFIED, Date.class, new Date());
			
			...	
		}
		
		private Object addItem(Object[] values) {...}
		private Object addChild(Object[] values, Object parentId) {...}
		private void setProperties(Item item, Object[] values) {...}
		private void addChildren(Object itemId) {...}
		private boolean removeChildrenRecursively(Object itemId) {...}
		
		@Override
		public boolean hasChildren(Object itemId) {...}
	}
    \end{lstlisting}
	\caption{Contenedor TreeGrid}
	\label{fig:jobContainer}
\end{figure}


\begin{figure}[!tb]
	\centering
	\begin{lstlisting}[language=Java]
	public class JobContainer
		...
		private Map<Object, Boolean> expandedNodes = new HashMap<>();
			
		@Override
		public void setCollapsed(Object itemId, boolean collapsed) {
			expandedNodes.put(itemId, !collapsed);	
			if (collapsed) {
				removeChildrenRecursively(itemId);
			} else {
				addChildren(itemId);
			}
		}
		
		@Override
		public boolean isCollapsed(Object itemId) {
			return !Boolean.TRUE.equals(expandedNodes.get(itemId));
		}
	}\end{lstlisting}
	\caption{Contenedor TreeGrid Collapsible}
	\label{fig:jobContainerCollapsible}
\end{figure}

\begin{figure}[!tb]
	\centering
	\begin{lstlisting}[language=Java]	
	public class JobContainer
	
		...
		@Override
		public int getDepth(Object itemId) {
			int depth = 0;
			while (!isRoot(itemId)) {
				depth ++;
				itemId = getParent(itemId);
			}
			return depth;
		}
	}\end{lstlisting}
	\caption{Contenedor TreeGrid Measurable}
	\label{fig:jobContainerMeasurable}
\end{figure}




%%==========================================================================%%
%% NOTA(Pablo): Describir lo que representa la figura que te he puesto      %%
%%              arriba                                                      %%
%%==========================================================================%%



%%==========================================================================%%
%% NOTA(Pablo): Decir esto y no decir nada es lo mismo. Te lo he encauzado  %%
%%    un poco mejor arriba.                                                 %%
%%==========================================================================%%
%% 
%% Echando un breve vistazo a la interacción cliente/servidor que facilita
%%  Vaadin, podemos explicar el código desde un punto de vista cliente y
%% servidor. El navegador, que realiza las funciones de cliente, recibe el 
%% conjunto de ficheros \emph{HTML},\emph{Javascript} y \emph{CSS} 
%% necesarios para formar la vista, este lado se encarga de cargar y 
%% visualizar dichos ficheros, así como, de recibir las acciones sobre la
%% vista para comunicarlas a la parte servidora, utilizando para ello 
%% llamadas \emph{AJAX}\cite{ajax}. Desde el punto de vista servidor,  
%% Vaadin recibe las llamadas y las trata para posteriormente recargar 
%% la vista enviando una respuesta al cliente.
%% 
%%
%% 
%% Extrapolando al ejemplo, la \emph{UI} se compilará para poder ser
%% visualizada por un navegador. El usuario al interactuar con la interfaz,
%%  enviará eventos a través del contenedor de datos, una vez finalizados 
%% los eventos, se volverá a recargar la vista para hacer visibles los
%%  cambios. De esta forma, en la parte cliente se almacenarán los ficheros 
%% \emph{HTML},\emph{Javascript} y \emph{CSS} mientras que en la parte 
%% servidora se encontrará toda la lógica.
%%
%%==========================================================================%%

Con estas indicaciones se ha creado un ejemplo sencillo de composición de elementos jerárquicos entre sí utilizando Vaadin, como podemos ver ha facilitado mucho su implementación respecto a una configuración basada en \emph{HTML} o \emph{Javascript}.


 			
	
	

	
	
	
	\clearpage
	
	\chapter{Process-Component}
	

	

Este fragmento del proyecto o componente se dedica exclusivamente al apartado gráfico, el cuál posee una sintaxis ya definida en el apartado del diseño arquitectónico.



Este componente es una herramienta que se dedica a proveer métodos para interactuar con él y además es capaz mediante escuchadores de avisar a los clientes que lo requieran sobre los eventos ocurridos sobre la interfaz gráfica.


Su implementación se centra en dos pilares centrales. Por una parte se ha realizado un proyecto Vaadin encargado de mantener el estado de la aplicación gráfica, y por otro lado un conjunto de ficheros Javascript implementados sobre GO.JS que se encargan de modificar el entorno gráfico mediante sentencias .
\begin{itemize}
	\item Proyecto Vaadin
	\subitem Este proyecto es el encargado de crear los metodos necesarios para interactuar desde el exterior con el esquema creado previamente. Además debe de permitir insertar escuchadores para los eventos proporcionados desde GO.JS, de forma que se pueda establecer dos direcciones de comunicaciones. Una desde el exterior con el proyecto Vaadin y este con el conector y directamente con el entorno gráfico de GO.JS, y otro desde interacción con los eventos (por parte del usuario) desde el fichero Javascript de configuración (explicado en el apartado posterior), con el conector y este con el estado, es decir, con el proyecto Vaadin.
	
	\item Ficheros Javascript
	\subitem Existe un primer fichero que permite configurar el esquema gráfico que se va a llevar a cabo (Procesos, Subprocesos, InputVariables, OutputVariables ...), así como todo el resto de propiedades gráficas, ademas de ser capaz de lanzar eventos preconfigurados.
	
	
	El segundo fichero esencial para el funcionamiento de esta estructura es el fichero conector. Este es el encargado de declarar y configurar todos los eventos que se pueden lanzar, además de ser el encargado de realizar todos los metodos CRUD \footnote{CRUD es el acrónimo de crear, leer, actualizar y borrar, en esete contexto significa el conjunto de métodos para poder realizar dichas acciones sobre los distintos elementos existentes.}necesarios para que puedan interactuar con las propiedades configuradas en el primer fichero citado previamente.
\end{itemize}

Esta imagen trata de  resumir el esquema de actuación que se lleva a cabo para la comunicación entre los distinto elementos del Process-Component.

\begin{figure}[H]
	\centering
	\includegraphics[scale=1.25]{schema.png}
	\caption{Estructuración en módulos}\label{fig:schema}
\end{figure}
	
	\clearpage
	
	\chapter{Integración del \emph{Process Editor} en LUCA}
	
Este capitulo describe el proceso de desarrollo e integración del componente gráfico \emph{Process-Component}. Se analizarán los aspectos del diseño y arquitectura a realizar así como las etapas de desarrollo del mismo.
	

El proyecto estará basado en una arquitectura en tres capas. La capa de servicios se comunicará con la capa de repositorio y con una serie de conectores que actuarán de intermediarios con recursos externos al sistema. Además, la capa de presentación utilizará el proyecto implementado con \emph{GoJS} como componente integrado en la interfaz (Figura~\ref{fig:arquitecturaLuca}).

\begin{figure}[H]
	\centering
	\includegraphics[width=\linewidth]{arquitecturaLuca.png}
	\caption{Nueva Arquitectura de LUCA}\label{fig:arquitecturaLuca}
\end{figure}


El primer paso para empezar a comprender el diseño y arquitectura de LUCA, es entender el patrón MVP\cite{mvp} utilizado.
A nivel de síntesis, el patrón MVP(\emph{Model View Presenter}) consta de tres capas. La capa del \emph{Modelo} es la encargada de albergar toda la lógica de negocio. La capa de la \emph{Vista} posee la capacidad de mostrar datos sobre las diferentes vistas. Por último, la capa de \emph{Presentación} realiza las labores de capa intermedia entre las citadas anteriormente y es capaz de conectar la interfaz gráfica con los daos.


Entrando en el territorio propio de las capas, \emph{LUCA} se compone de tres capas bien diferenciadas:

\begin{enumerate}
	\item Capa de Presentación \subitem Esta capa es la encargada de mostrar los datos sobre la interfaz.
	\item Capa de Servicio \subitem Esta capa se ocupa de obtener los deatos de las diversas fuentes o recursos, ya bien sea comunicándose con la capa de repositorio, o con los diversos conectores hacia fuentes de datos externas. Además, esta capa se divide realmente en dos capas, la mencionada anteriormente, y otra por encima llamada \emph{Controladora} que controla todos los aspectos derivados de la seguridad en el acceso a datos.
	\item Capa de repositorio \subitem Esta capa es la encargada de comunicarse con la base de datos, así como, de implementar entre otras funciones, los filtros de petición de datos (un ejemplo de filtro sería obtener los datos de los procesos que se encuentren en un estado determinado). Además, se implementa con \emph{Spring Data}\cite{jpa}, que se encarga de asegurar la correcta comunicación con la base de datos.
\end{enumerate}




	
	\clearpage
	
	\chapter{Sumario, Conclusiones y Trabajos Futuros}

\section{Sumario}

El presente documento ha descrito el trabajo realizado para integrar el concepto de \emph{proceso} en LUCA. Gracias a este trabajo es posible ejecutar en LUCA procesos de recuperación de la información que requieran de la ejecución de varias consultas encadenadas, donde las salidas de ciertas consults sirvan como entradas de otras. 

Dado que el trabajo presentado se enmarca dentro del proyecto LUCA, en primer lugar se ha descrito dicho proyecto en profundidad; detallando su objetivo, que es el de proporcionar mecanismos de acceso uniforme a fuentes de datos heterogéneas, y sus limitaciones. Entre dichas limitaciones se encontraba la carencia de un soporte adecuado para ejecutar procesos de recuperación de la información que precisasen de la ejecución de varias consultadas encadenadas. Antes del desarrollo del proyecto, el usuario debía ejecutar cada consulta del conjunto de consultas encadenadas manualmente, guardar sus resultados y manipularlos adecuadamente para posteriormente poder usarlos como entradas para otras consultas. 

Para paliar esta deficiencia, los directores de LUCA habían elaborado una especificación de requisitos para un nuevo módulo de LUCA que soportase el concepto de proceso, incluyendo tanto su especificación, la definición del proceso, como su ejecución. Dado que la especificación de requisitos ya estaba hecha, y la arquitectura definida, este proyecto fin de carrera se circunscribe a las fase de diseño detallado, implementación y pruebas. 

Una vez definidas la motivación y el alcance del proyecto, se describieron una serie de tecnologías necesarias para entender su funcionamiento. En primer lugar se explicó el funcionamiento del framework \emph{GoJS} cuyo objetivo es facilitar la creación de editores gráficos en Javascript, el cual fue utilizado en el desarrollo del proyecto para la creación del editor gráfico de procesos. A continuación, se describió la arquitectura de LUCA, que es el proyecto raíz donde se integra el módulo desarrollado en este Trabajo Fin de Grado. Dado que LUCA está construido sobre \emph{Vaadin}, un framework para el desarrollo de aplicaciones web enriquecidas desde código Java, y que \emph{Vaadin} sigue el patrón \emph{Model-View-Presenter}, antes de describir la arquitectura de LUCA se introdujeron ambos elementos. 

Finalmente, se describieron las diversas fases del desarrollo del presente proyecto. El proyecto se ha desarrollado de manera iterativa, moviéndonos en cada iteración desde la capa de presentación a la capa de persistencia. 

\section{Resultados}

Gracias a este proyecto, LUCA soporta actualmente el concepto de \emph{proceso}. Más concretamente, gracias a este proyecto, LUCA soporta ahora las siguientes funcionalidades. 

\begin{enumerate}
	\item Especificación gráfica de \emph{procesos}, mediante la definición de las consultas que componen dicho proceso y su interconexión.
	\item Alteración de la ejecución del flujo de ejecución de un proceso en función de los valores de las salidas.
    \item Comprobación de la corrección de la conexión entre salidas y entradas de las consultas que componen un proceso.
    \item Ejecución de procesos.
	\item Ejecución paso a paso de los procesos.
	\item Exportar los resultados de los procesos.
\end{enumerate}

\section{Conclusiones}

%% Detalla qué conocimientos del Grado te han sido especialmente útiles.

La experiencia durante el desarrollo del proyecto ha sido más que satisfactoria, ya que los conocimientos que he aprendido a lo largo del grado, tales como los patrones de diseño, las diferentes arquitecturas software o el despliegue de servicios entre muchos otros , han sido indispensables para poder llevarlo a cabo. Además, debido al uso de nuevos frameworks y herramientas, he sido capaz de ampliar mis conocimientos, a los que he incorporado la utilización de frameworks como \emph{Vaadin} y \emph{GoJS}.

La oportunidad de llevar a cabo el proyecto en la empresa CIC Consulting Informático, me ha permitido adquirir un hábito de trabajo, así como de responsabilidades y fijaciones a horarios que no se puede aprender de ninguna otra forma. Además, el apoyo recibido por el equipo de LUCA ha sido excepcional y he podido aprender mucho con ellos.

\section{Trabajos Futuros}

Con el proyecto integrado en LUCA, se espera poder realizar el trabajo realizado en futuras versiones del producto. Concretamente, se plantea la posibilidad de soportar procesos anidados, es decir, la capacidad de utilizar procesos ya definidos como si fuesen consultas simples para la definición de un nuevo proceso. Es decir, permitir que los procesos puedan contener procesos. También se plantea la posibilidad de crear consultas desde una consola gráfica, utilizando el editor gráfico, ya que en algunos casos este estilo de especificación puede resultar más conveniente.


		
	
	\clearpage

	


	
	\bibliographystyle{acm}
	\bibliography{Bibliografia}
	

\end{document}


