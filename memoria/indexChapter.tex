\chapter*{Agradecimientos}
	Me gustaría dar agradecimientos a mi familia y facultad, ya que sin ellos esto no habría sido posible nada de esto.
	
	 Es importante agradecer también a CIC Consulting Informático por permitirme la oportunidad de realizar el desarrollo del proyecto en su empresa, sin olvidarme de mis compañeros de LUCA, que han sido un gran apoyo durante el mismo.
	
	 Para finalizar, me gustaría gradecer a mi mentor Pablo, por guiarme durante el desarrollo del proyecto con eficacia y ayudarme a afrontar este trabajo de fin de grado.

	
\chapter*{Resumen}
	Las empresas actuales no utilizan un único sistema de información que de soporte a sus procesos de trabajo, sino un  ecosistema de sistemas información que dan soporte a diferentes procesos de negocio ejecutados dentro de dicha organización. Como consecuencia de esta nueva situación, cuando un usuario	quiere obtener una información concreta cuyos datos residen en varios de estos
	sistemas, necesita acceder a cada uno de estos sistemas, extraer de cada sistema la información que precisa, filtrarla y unificarla para finalmente	obtener los datos requeridos.
	
	
	Por ejemplo, una tienda de electrodomésticos podría tener sistemas informáticos diferentes para el departamento de atención al cliente, para el departamento técnico de postventa y para el departamento de compras y adquisiciones. Por tanto, para conocer el estado actual de una reparación, podríamos necesitar:
		\begin{itemize}
			\item  Acceder al primer sistema para obtener el identificador de la incidencia y en qué fase de su gestión se encuentra.
			\item  Comprobado que la incidencia está actualmente en reparación, recuperaríamos otro sistema el estado detallado de la reparación, comprobando que está a la espera de una pieza.
			\item Finalmente accederíamos al sistema de compra y adquisiciones para comprobar cuando está prevista la entrega de dicha pieza. Los sistemas de almacenamiento de la información pueden ser diversos, incluyendo desde un servicio web, una base de datos relacional, un repositorio de ficheros accesible vía FTP o una base de datos NoSQL.
		\end{itemize}
	
	
El objetivo de este proyecto es facilitar dicho proceso de composición al usuario mediante el desarrollo de un mecanismo gráfico para la especificación de estos procesos de composición de consultas.

	
 \textbf{Palabras clave}
 Proceso, Consulta, Vaadin, Go.JS, MVP, LUCA.
	
\chapter*{Preface}

Currently, companies rarely use a single software software system to support their business processes. Instead, several software systems, typically one per department or business unit, are used.  As a consequence of this situation, when a user wants to retrieve a piece of information whose data are stored across these
systems, she would need to access each one of these systems; extract the required information from each system; filter this information; and, finally, unify it to obtain the appropriate data.
	
For example, an appliance store could have different information systems: one for the customer service department; one for technical assistance department; and one for the purchasing and procurement department. Therefore, to know the current status of a repair, it be might needed:

\begin{itemize}

    \item To access the customer service system to get the identifier of the identifier of the repair order and to know at what stage of the repair process this order is.
    \item After checking that order is currently being processed, it might be needed to access the information system of the technical assistance department to find the details of the repair status. In this case, it might be discovered that the technicians are waiting for a component to complete the repair.
	\item Finally, to know when this component is expected to be delivered, we would need to access the information system of the purchase and acquisition department.
	\end{itemize}

Each one of these systems may store its data into a different kind of system, ranging from a web service to a relational database, a file repository accessible via FTP or a NoSQL database, among other options. The objective of this project is to develop a graphical editor that support the specification of this complex information retrieval processes, so that business people can easily recover this information by simply executing this processes, and abstracting from the details of the sources for these data.


	\textbf{Keywords}:
	Process, Query, Vaadin, Go.JS, MVP, LUCA. 